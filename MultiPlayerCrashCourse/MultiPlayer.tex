\documentclass[CJKmath=true,10pt]{MyBook}


\usepackage{graphicx,float}

\newcommand{\cgc}[3][]{
\begin{figure}[H]
\centering
\includegraphics[#1]{#2}
\caption{#3}
\end{figure}}

\begin{document}

%\frontmatter\tableofcontents
\mainmatter
\setchapterimage{./images/055}
\chapter{Actor Replicates}
\section{Actor 复制核心属性}
Unreal Engine 中的 Actor 支持网络复制,主要通过以下三个属性控制:
\begin{enumerate}
	\item \textbf{NetLoadOnClient}: 决定 Actor 是否在地图加载时于客户端加载。若为 \texttt{false},Actor 仅存在于服务器。
	\item \textbf{Replicates}: 标记 Actor 是否进行网络复制。若为 \texttt{true},它会强制开启 \texttt{NetLoadOnClient},并允许变量进行网络同步。
	\item \textbf{ReplicateMovement}: 独立控制移动数据的复制。即使 \texttt{Replicates} 为 \texttt{true},若此项为 \texttt{false},Actor 的位置和旋转变动也不会同步到客户端。
\end{enumerate}

\subsection{C++ 中的实现细节}
在 C++类 的构造函数中初始化这些属性时需遵循特定规则:
\begin{itemize}
	\item \textbf{bNetLoadOnClient}: 直接赋值 \ci{bNetLoadOnClient = true;}。
	\item \textbf{bReplicates}: 直接赋值 \ci{bReplicates = true;}。避免使用 \ci{SetReplicates()} 函数,因为它是为运行时设计的,在构造函数中调用会触发警告。
	\item \textbf{ReplicateMovement}: 由于成员变量是私有的,且 \ci{SetReplicateMovement()} 是虚函数(\emph{不应在构造函数中调用虚函数}),应使用非虚函数 \ci{SetReplicatingMovement(true);} 来设置。
\end{itemize}

\subsection{橡皮筋效应 (Rubber Banding)}
当服务器与客户端的数据不同步时,会出现“橡皮筋效应”:
\begin{itemize}
	\item \textbf{现象}: 玩家尝试移动,但被服务器强制拉回原位。
	\item \textbf{成因}: 例如客户端看不到某个阻挡物(\ci{NetLoadOnClient = false}),尝试穿过它;或者物体在两端位置不一致。服务器检测到非法移动并进行修正。
	\item \textbf{解决}: 确保阻挡玩家的物体在客户端加载,并且开启移动复制以保持位置同步。
\end{itemize}

\subsection{物理模拟与 Root Component}
对于启用了物理模拟(Simulate Physics)的组件,复制行为主要依赖于它是否为 Root Component:
\begin{itemize}
	\item \textbf{问题}: 如果一个 Static Mesh 模拟物理但\textbf{不是} 根组件,\ci{ReplicateMovement} 只会同步 Root 的位置,导致 Mesh 在客户端与服务器的位置不一致(Root 不动,Mesh 掉落或滚动)。
	\item \textbf{解决方案}: 将模拟物理的组件设置为 Actor 的 \textbf{Root Component}。这样 \ci{ReplicateMovement} 就能正确同步物理模拟后的实际位置。
\end{itemize}

\section{权限 (Authority) 与 网络角色 (Net Role)}

在多人游戏中,Authority 和 Net Role 是决定 Actor 由谁控制以及如何在网络中同步的核心概念。

\subsubsection{权限 (Authority)}
Authority 定义了哪个机器“拥有”该 Actor 并负责其逻辑判断。
\begin{itemize}
	\item \textbf{服务器权限}: 大多数 Replicated Actor(在服务器生成或放置在地图中)在服务器上具有 Authority。服务器负责权威逻辑。
	\item \textbf{客户端权限}: 仅当 Actor 是由客户端生成(Spawn)且不进行网络复制时,该客户端拥有 Authority。
	\item \textbf{HasAuthority()}: 可以调用此函数检查当前 Actor 实例是否拥有权限(通常用于判断当前代码是否运行在服务器上)。
\end{itemize}

\subsubsection{网络角色 (Net Role)}
每个 Actor 在不同机器上都有特定的网络角色 (\ci{ENetRole}),分为 \textbf{Local Role}(本地角色)和 \textbf{Remote Role}(远程角色)。

\textbf{常见的 Local Net Role 类型}:
\begin{itemize}
	\item \textbf{Authority}: 拥有该 Actor 的机器(通常是服务器)。
	\item \textbf{Autonomous Proxy}: 在客户端机器上,由该客户端 \textbf{本地控制} 的 Character 或 Pawn(例如玩家自己的角色)。它有权进行预测性移动。
	\item \textbf{Simulated Proxy}: 在客户端机器上,由服务器或其他客户端控制的 Character,或者是其他非玩家控制的 Replicated Actor(如移动的方块)。客户端只负责根据服务器数据模拟其状态。
\end{itemize}

\subsubsection{Remote Role (远程角色)}
\textbf{Remote Role} 指的是相对于当前机器,该 Actor 在远程机器上的角色。
\begin{itemize}
	\item 在 \textbf{客户端} 上检查 \ci{GetRemoteRole()},通常返回 \ci{Authority}(因为服务器拥有它)。
	\item 在 \textbf{服务器} 上检查 \ci{GetRemoteRole()},通常返回 \ci{SimulatedProxy}。即使该 Actor 在某个客户端是 \ci{Autonomous Proxy},服务器端查询 Remote Role 时也可能显示为 \ci{Simulated Proxy}\footnote{这是一个常见的误区,需注意 \ci{GetRemoteRole} 的行为}。
\end{itemize}

\subsubsection{调试与工具}
可以通过创建蓝图函数库(Blueprint Function Library)来辅助调试:
\begin{itemize}
	\item 使用 \ci{GetLocalRole()} 获取当前实例的角色。
	\item 使用 \ci{GetRemoteRole()} 获取远程机器上的角色。
	\item 将 Enum 转换为字符串输出日志,可以清晰地看到同一个 Actor 在 Server(Authority)、Owner Client(Autonomous)和其他 Client(Simulated)上的不同状态。
\end{itemize}

\section{多人游戏中的附件 (Attachment)}

在多人游戏中,将一个 Actor 或 Component 附加到另一个 Actor 上(例如给角色装备护甲或武器)时,需要特别注意网络复制(Replication)的规则,以确保服务器和客户端的表现一致。

\subsection{Actor 级别的附件 (Actor Attachment)}
将一个完整的 Actor 附加到另一个 Actor 上(例如通过 \texttt{AttachToActor}):
\begin{itemize}
	\item \textbf{前提条件}: 被附加的 Actor 必须标记为 \textbf{Replicates}(\texttt{SetReplicates(true)})。
	\item \textbf{行为}: 如果 Actor 是 Replicated 的,并且在服务器上执行了 \texttt{AttachToActor},该附件关系会自动复制到所有客户端。
	\item \textbf{流程}:
	\begin{enumerate}
		\item 在服务器上检测到触发事件(如 Overlap)。
		\item 检查权限 \texttt{HasAuthority()},确保仅在服务器执行。
		\item 调用 \texttt{AttachToActor}。
		\item 引擎自动将此附件关系由服务器同步至客户端。
	\end{enumerate}
\end{itemize}
从源代码分析:Actor级别的挂接(attachment)只会将\ci{Root}组件挂接到目标Actor的\ci{Root}组件。
\subsection{Component 级别的附件 (Component Attachment)}
现在对代码是否位于服务器进行检测,如果位于服务器则将组件挂接到目标Actor指定的位置:
\begin{code}
if (!HasAuthority()) return;

// AttachToActor只挂接到OtherActor的根组件上,而Character的根组件是CapsuleComponent,不是想要的Mesh组件
//AttachToActor(OtherActor, FAttachmentTransformRules::SnapToTargetNotIncludingScale, "ArmorSocket");
ACharacter* Character = Cast<ACharacter>(OtherActor);
if (!Character) return;
SphereMesh->AttachToComponent(Character->GetMesh(), FAttachmentTransformRules::SnapToTargetNotIncludingScale,
"ArmorSocket");
\end{code}

运行后,无论Armor 是否设置\ci{Replicates},客户端与服务端不能同步,表现为:无论玩家控制客户端还是服务端拾取Armor,均只能在服务端上显示Armor挂接到指定socket。

将一个 Actor 内部的某个 Component 附加到另一个 Actor 的 Component 上(通过 \ci{AttachToComponent},常用于绑定到骨骼插槽 Socket):
\begin{itemize}
	\item \textbf{关键点}: 仅仅 Actor 是 Replicated 的还不够。如果附件操作是针对 Component 的,那么该 Component 自身必须被标记为 \textbf{*Replicated Component}。
	\item \textbf{设置方法}: 在 C++ 构造函数中,对该 Component 调用 \texttt{SetIsReplicated(true)}。
	\item \textbf{行为}:
	\begin{itemize}
		\item 如果 Component 未设置为 Replicated:在服务器执行 \texttt{AttachToComponent} 后,只有服务器会看到附件变化,客户端不会更新。
		\item 如果 Component 设置为 Replicated:服务器上的附件操作会自动同步到客户端,客户端也能看到Component附着在正确的位置(如角色的头部或脚部 Socket)。
	\end{itemize}
\end{itemize}

\subsection{最佳实践与性能提示}
虽然将 Component 设置为 Replicated 可以解决附件同步问题,但这会增加网络带宽开销(Overhead)。
\begin{itemize}
	\item 每个 Replicated Component 都会占用网络资源。
	\item 对于复杂的装备系统,通常不仅仅依赖 Component Replication,而是倾向于使用变量复制(Replicated Variables)配合 \texttt{OnRep} 函数。即复制一个状态变量,当客户端收到变量更新时,在本地执行附加逻辑,从而节省带宽。
\end{itemize}
\end{document}